\providecommand{\main}{..}
\documentclass[../main.tex]{subfiles}

\title{Nam derivation}
\author{\begin{telugu}హృదయ్ దీపక్ మల్లుభొట్ల\end{telugu}}
% want empty date
\predate{}
\date{}
\postdate{}

\begin{document}

	% !TeX spellcheck = en_GB
	\onlyinsubfile{\maketitle}
	\section{Nam Conductivity Derivation} \label{sec:Nam}
	Let's summarise here the derivation given by Nam~\autocite{Nam1967}.
	He begins with
	\begin{equation}
		J_\mu(x) = - \frac{1}{4\pi}\int \dd[4]{x'} \underline{K}_{\mu\nu}(x, x') A^\nu(x'),
	\end{equation}
	or, in frequency space\footnote{
		As an aside, we can compare this with equation~(3.174) in~\cite{Giuliani2005}:
		\begin{equation*}
			\vec{\jmath}_{L(T)}(q, \omega) = \frac{e}{c} \chi_{L(T)}(q, \omega) \vec{A}_{L(T)}(q, \omega),
		\end{equation*}
		which tells us that we're indeed working with the correct response functions.
	},
	\begin{equation}
		J_\mu(q, \omega) = - \frac{1}{4\pi} K_{\mu\nu}(q, \omega) A^\nu(q, \omega),
	\end{equation}

	Nam shows:
	\begin{itemize}
		\item that there's an explicit value $K(0, 0)$ such that
			\begin{equation}
				\lim_{q \rightarrow 0} \lim_{\omega \rightarrow 0} K(q, \omega) = \lim_{\omega \rightarrow 0} \lim_{q \rightarrow 0} K(q, \omega) = K(0, 0),
			\end{equation}
		\item the explicit form of the response function $K$
		\item the way the response function breaks into normal and superconducting parts.
	\end{itemize}

	\subsection{Diamagnetic / Paramagnetic Decomposition} \label{subsec:diaparamag}

	Nam decomposes the response function into a diamagnetic and paramagnetic part:\todo{finish this idea}
	\begin{equation}
		K_{\mu\nu} = K_{\mu\nu}^p + K_{\mu\nu}^d
	\end{equation}
	\subsection{Response functions} \label{subsec:}

	Defining a correlation function
	\begin{equation}
		P_{\mu\nu}(x, x') = 4 \pi i \ev{T j_\mu^p(x) j_\nu^p(x')} \label{eq:corrf}
	\end{equation}
	where $j^p$ is the paramagnetic part of the current.\todo{Is there a ``clean'' way to define this?
		Possible in terms of just looking at the decomposition of $j$ operator, but prefer something more declarative}

	Nam says that $P_{\mu\nu}(q, \omega)$ has the same imaginary and real parts as $K_{\mu\nu}^p$, which is convenient.\todo{fd theorem}
	We work in the spinor representation (see Nambu~\autocite{Nambu1960}) with
	\begin{align}
		\Psi_k^{\dagger}(t) &= \begin{pmatrix}
							\psi_{k \uparrow}^\dagger(t) & \psi_{-k \downarrow}(t)
			\end{pmatrix} \\
		\Psi_k(t) &= \begin{pmatrix}
								\psi_{k \uparrow}(t) \\
								\psi_{-k \downarrow}^\dagger(t)
		\end{pmatrix}
	\end{align}
	The paramagnetic part of the current operator is defined via
	\begin{equation}
		j_\mu^p(\vec{q}, t) = e \sum_k \Psi_k^{\dagger}(t) \gamma_\mu(\vec{k}, \vec{k + q}) \Psi_{k + q}(t), \label{eq:current}
	\end{equation}
	where
	\begin{align}
		\gamma_\mu(\vec{k}, \vec{k + q}) &= v_\mu \left(1 - \delta_{\mu, 0} \right) + \tau_3 \delta_{\mu, 0} \\
		\tau_3 &= \begin{pmatrix}
					 1 & 0 \\
					 0 & -1
				\end{pmatrix} \\
		v_\mu &= \left( \nabla_k \epsilon_k \right)_\mu + \frac{1}{2}\left( \nabla_{k + q} \epsilon_{k + q} - \nabla_k \epsilon_k \right)_\mu
	\end{align}
	For future edification, we can write down that
	\begin{align}
		\gamma_{\mu = 0} &= \tau_3 \\
		\gamma_{\mu \neq 0} &= v_\mu I
	\end{align}
	where $I$ is the identity on the spin indices.

	We can insert~\eqref{eq:current} into~\eqref{eq:corrf},
	\begin{align}
		P_{\mu\nu}(x, x') &= 4 \pi i \ev{T j_\mu^p(x) j_\nu^p(x')} \\
		P_{\mu\nu}(x, x') &= \int e^{-i q x'} 4 \pi i e^2 \ev{T \sum_{k, k'} \Psi_k^{\dagger}(t) \gamma_\mu(\vec{k}, \vec{k + q}) \Psi_{k + q}(t) \Psi_k'^{\dagger}(t') \gamma_\nu(\vec{k'}, \vec{k' + q}) \Psi_{k' + q}(t')} \\
	\end{align}

\end{document}
