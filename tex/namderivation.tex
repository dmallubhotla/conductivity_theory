\providecommand{\main}{..}
\documentclass[../main.tex]{subfiles}

\title{Nam derivation}
\author{\begin{telugu}హృదయ్ దీపక్ మల్లుభొట్ల\end{telugu}}
% want empty date
\predate{}
\date{}
\postdate{}

\begin{document}

	% !TeX spellcheck = en_GB
	\onlyinsubfile{\maketitle}
	\section{Nam Conductivity Derivation} \label{sec:Nam}
	Let's summarise here the derivation given by Nam~\autocite{Nam1967}.
	He begins with
	\begin{equation}
		J_\mu(x) = - \frac{1}{4\pi}\int \dd[4]{x'} \underline{K}_{\mu\nu}(x, x') A^\nu(x'),
	\end{equation}
	or in frequency space\footnote{
		As an aside, we can compare this with equation~(3.174) in~\autocite{Giuliani2005}:
		\begin{equation*}
			\vec{\jmath}_{L(T)}(q, \omega) = \frac{e}{c} \chi_{L(T)}(q, \omega) \vec{A}_{L(T)}(q, \omega),
		\end{equation*}
		which tells us that we're indeed working with the correct response functions.
	},
	\begin{equation}
		J_\mu(q, \omega) = - \frac{1}{4\pi} K_{\mu\nu}(q, \omega) A^\nu(q, \omega),
	\end{equation}

	Nam shows:
	\begin{itemize}
		\item We can show that there's an explicit value $K(0, 0)$ such that
			\begin{equation}
				\lim_{q \rightarrow 0} \lim_{\omega \rightarrow 0} K(q, \omega) = \lim_{\omega \rightarrow 0} \lim_{q \rightarrow 0} K(q, \omega) = K(0, 0)
			\end{equation}
	\end{itemize}

\end{document}
