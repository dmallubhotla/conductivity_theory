\providecommand{\main}{..}
\documentclass[../main.tex]{subfiles}

\title{Tinkham Electrodynamic response}
\author{\begin{telugu}హృదయ్ దీపక్ మల్లుభొట్ల\end{telugu}}
% want empty date
\predate{}
\date{}
\postdate{}

\begin{document}

	% !TeX spellcheck = en_GB
	\onlyinsubfile{\maketitle}
	\section{Tinkham's Superconductor response} \label{sec:TinkhamResponse}

	We'll talk about Tinkham's derivation\supercite{Tinkham}, which shows the same $\sigma\sim\frac{1}{q}$ dependence that's caused us issues.

	\subsection{Two-Fluid Model Drude Analysis} \label{subsec:twofluid}

	Summarising Drude here, for any particular channel parameterised by some scattering time $\tau_i$, where $i$ indexes the channel,
	\begin{align}
		\sigma_i(\omega) &\equiv \sigma_{1i} - i \sigma_{2i} \\
		\sigma_i(\omega) &= \frac{\flatfrac{n_i e^2 \tau_i}{m}}{1 + i \omega \tau_i} \label{eq:tinkhamdrude}
	\end{align}
	Note that effectively this is for some reason the complex conjugate of the Drude conductivity we defined earlier.\todo{why?}
	For some two-fluid model, we can allow two channels $i \in \{n, s\}$, and describe the superconducting channel by sending $\tau_s \rightarrow \infty$.
	As Tinkham describes, this leads to a delta function, and we get for the total conductivity
	\begin{align}
		\sigma_1(\omega) &= \frac{\pi n_s e^2}{2m} \delta(\omega) + \frac{n_n e^2 \tau_n}{m} \\
		\sigma_2(\omega) &= \frac{n_s e^2}{m \omega}.
	\end{align}
	This by itself doesn't mean a ton, as this two-fluid model isn't expected to be a spectacular approximation.
	It does, however, point to a couple features we should expect for the superconductivity, which is that there still is dissipative flow for any non-zero frequency (where dissipative flow is controlled by $\sigma_1$).
	Also, we expect some specific signs for the conductivity's real and imaginary parts, and thus specific signs for the dielectric constant.
	With $\epsilon \sim i \sigma$, we should expect that $\epsilon$ has a positive real and imaginary part, if $\sigma_1$ and $\sigma_2$ are both positive.
	We don't end up seeing this in other forms, so we should investigate this.\todo{make sure the signs are consistent.}

\end{document}
